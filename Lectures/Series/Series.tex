\documentclass[a4paper,12pt]{article}
\usepackage{amsmath, amssymb}
\usepackage{xcolor}
\usepackage{ulem}        % underline
\usepackage{tcolorbox}   % colored boxes (if needed)
\usepackage{pifont}
\usepackage{xcolor}

\begin{document}

\begin{center}
\textbf{Mathématiques : CM}
\end{center}

\[
\sum_{\,n\,=\,0}^{+\infty} u_n \in \mathbb{R} \quad \text{if } (u_n) \text{ converges } 
\Leftrightarrow u_0 + u_1 + u_2 + \dots 
\Leftrightarrow \lim_{n \to \infty} \sum_{\,k\,=\,0}^n u_k
\]

\[
\sum_{\,k\, \geq\, 0} u_k \in \mathbb{R}^{\mathbb{N}} 
\quad \Leftrightarrow \quad \sum_{\,k\, =\, 0} ^ {n} u_k = S_n \in \mathbb{R}
\]

\[
\text{The notation} \sum u_n\quad \text{if we don’t know or don’t care about the first terms.}
\]

\noindent
The two previous ones are named \textcolor{red}{\underline{Series}} as they are equal to a sequence.

\vspace{0.5cm}

\begin{center}
\Large \textcolor{red}{\section*{Series}}
\end{center}

\textcolor{red}{\subsection*{Definitions:}}\bigskip

\textcolor{red}{\subsubsection*{Convergence and divergence:}}\bigskip

\text{Let $(u_n) \in \mathbb{R}^{\mathbb{N}}$ be a sequence and $(S_n) \in \mathbb{R}^{\mathbb{N}}$ the sequence of its partial sums, i.e.}
\[
S_n = \sum_{\,k\,=\,0}^n u_k.
\]

We call \underline{series} of the sequence $(u_n)$ the general term $S_n$ and denote
\[
\sum u_n.
\]

\text{We say \(\sum u_n\) is \textcolor{red}{convergent} if and only if $(S_n)$ converges, else \(\sum u_n\) \textcolor{red}{diverges}.}

\vspace{0.5cm}

\textbf{Example:} \quad 
\(\sum q^n\) : geometric series of general term \(q^n\).

\vspace{0.5cm}

\text{Let $(S_n) \in \mathbb{R}^{\mathbb{N}}$ (the sequence of partial sums). We have:}\bigskip

\text{For $n \in \mathbb{N}$,}
\[
S_n = \sum_{\,k\,=\,0}^n q^k = 1 + q + q^2 + \dots + q^n.
\]

\[
\Rightarrow q \neq 1 \quad \Rightarrow \quad 
S_n = \frac{1 - q^{n+1}}{1-q}.
\]

\[
\lim_{\,n\, \to\, +\infty} S_n =
\begin{cases}
\frac{1}{1-q}, & \text{if } |q| < 1, \\[6pt]
\infty, & \text{if } |q| \geq 1.
\end{cases}
\]

\[
\Rightarrow \text{diverges if else.}
\]

\begin{tcolorbox}[colback=red!5!white,colframe=red!100!black,title=\textcolor{black}{Conclusion}]
$\sum q^n$ \textcolor{red}{converges when $|q| < 1$} with
\[
\sum_{\,n\,=\,0}^{+\infty} q^n = \frac{1}{1-q},
\]
and \textcolor{red}{diverges when $|q| \geq 1$}.
\end{tcolorbox}

---

\textbf{Proposition:} \bigskip


\text{Let $\sum u_n$ and $\sum v_n$ be two numerical series and $\lambda \in \mathbb{R}$.\\ \\
Then we have:}
\bigskip

\begin{enumerate}
  \item If $\sum u_n$ and $\sum v_n$ converge, then $\sum (u_n + v_n)$ converges.
  \item If $\sum u_n$ converges, then $\sum \lambda u_n$ converges.
  \item If $\sum u_n$ converges and $\sum v_n$ diverges, then $\sum (u_n+v_n)$ diverges.
  \item If $\sum u_n$ diverges and $\sum v_n$ diverges, then \underline{we can’t tell anything}.
\end{enumerate}

\newpage
\textcolor{red}{\subsubsection*{Sum and Remainders of a Series}} \bigskip

\text{Let $\sum u_n$ be a numerical series. We call:} \bigskip

\text{\underline{Sum of $\sum u_n$} the following limit \textcolor{red}{(in case of convergence)}:}
  \[
  \lim_{\,n \,\to\, +\infty} \sum_{\,k\,=\,0}^n u_k = S_n,
  \]
  \text{where $S_n$ is the \textcolor{red}{sum of a convergent series}.}

\[ \text{Example :} \sum_ {} {q^n} \text{ converges if and only if } |q| < 1 \quad \text{and in that case:}
\]

\[
S_n = \lim_{\,n\, \to\, +\infty} S_n = \frac{1}{1-q}
\]


\textcolor{red}{\underline{Necessary condition of convergence:}}

\[
\sum u_n \text{ convergent } \Rightarrow \lim_{\,n \,\to\, +\infty} u_n = 0.
\]

\begin{tcolorbox}[colback=red!5!white,colframe=red!75!black]
\textbf{Contrapositive:}  
If $\underset{\,n \,\to\, +\infty}{\text{lim}} u_n \neq  0$ $\Rightarrow$ $\sum u_n$ \textcolor{red}{diverges}.
\end{tcolorbox}

\textcolor{red}{\subsection*{Positive Term Series:}}\bigskip

\text{We call \textcolor{red}{PTS (positive term series)} any series $\sum u_n$ such that $\forall$ n $\in$ $\mathbb{N}$ \; $u_n \geq 0 $. \\}\bigskip

\textbf{Proposition:} \text{Let $\sum u_k$ be a PTS and $(S_n)$ its sequence of partial sums. } 

\[
\text{Then: } \sum u_k \text{ is convergent } \Leftrightarrow (S_n) \text{ is bounded}.
\]
\\
Let $(u_n)$, $(v_n)$ be two positive sequences such that: $\forall \,{n} \in \mathbb{N}$, 0 $\leq$ $u_n$ $\leq$ $v_n$,\\
\\ Then :
\[
\begin{cases}

\sum v_n \text{ is convergent } \Rightarrow \sum u_n \text{ is convergent } \\[6pt]

\sum u_n \text{ is divergent } \Rightarrow \sum v_n \text{ is divergent }

\end{cases}
\]
\text{Ex :} \\
\[
\sum \frac{1}{n(sin(n))} = u_n
\]
\bigskip
\[
\forall \, n \in \mathbb{N^*}, 0 \leq |sin(n)| \leq 1 \Rightarrow u_n \ge \frac{1}{n} \text{ but} \sum \frac{1}{n} \text{ diverges} \Rightarrow \sum u_n \text{ diverges}
\]
\bigskip
\subsubsection*{Riemann Series Theorem}

\textcolor{red}{Definition :} \\

\text{We call \textcolor{red}{Riemann series} any series of general term :}
\[
\textcolor{red}{\frac{1}{n^\alpha}, \alpha \in \mathbb{R}}
\]

\text{Ex :} \\
\[
    \sum \frac{1}{n} , \sum \frac{1}{n^2}, \sum \frac{1}{n^\frac{9}{2}}
\] 
\\

\textcolor{red}{Riemann theorem :}\\
\[
\forall \,\alpha \in \mathbb{R}, \textcolor{red}{\sum{\frac{1}{n^\alpha}}} \textcolor{red}{   converges \Leftrightarrow \alpha > 1} 
\]
\\
\text{Ex :} 
\[
    u_n = \frac{2 + cos(n)}{n^4}, \forall n\in \mathbb{N^*}; 0 \leq \frac{2 + cos(n)}{n^4}\leq \frac{3}{n^4}
\] \\

\[
\text{As } \sum \frac{2 + cos(n)}{n^4} \text { is a PTS :}
\] \\

\[
\begin{cases}
    \sum u_n \text{ is a PTS}\\
    \sum \frac{1}{n^4} \text{ converges (Riemann } \alpha = 4 > 1\text{)} \Rightarrow \frac{3}{n^4} \text{ converges}
\end{cases}
\] \\

\text{Thus :} \[
\sum \frac{2 + cos(n)}{n^4} \text{ converges}
\] 
\\
\textcolor{red}{\subsubsection*{Proposition :}}
{\textcolor{red}{\ding{172}}}
\[
\text{Let (} u_n \text{) and (} v_n\text{) two sequences such that :}
\] \\
\textcolor{red}{\[
u_n = u_n = o(v_n) \quad(\frac{u_n}{v_n} \to 0) 
\]}\\
\text{Then :} 
\textcolor{red}{\[
\sum v_n \text{ converges} \Rightarrow u_n \text { converges}
\]}\\
{\textcolor{red}{\ding{173}}}\\
\textcolor{red}{\[\text{if } u_n \underset{\, +\infty}{\sim} v_n \text{ then :} \sum u_n \text{ same nature as } \sum v_n
\] }\\
{\textcolor{red}{\ding{174}}}\\
\[
\text{if } u_n = O(v_n) \text{ then :}
\] 
\textcolor{red}{\[
\sum v_n \text{ converges} \Rightarrow u_n \text { converges}
\] }\\
{\ding{172}}\\

\text{Ex :}\\
\[
\sum e^{-\sqrt{n}} \text{ (PTS)}
\]\\
\[
n^2 \times e^{-\sqrt{n}}  \underset{\,n \,\to\, +\infty}{\to} 0
\] \\
\[
\frac{e^{-\sqrt{n}}}{\frac{1}{n^2}} \Rightarrow e^{-\sqrt{n}} = o(\frac{1}{n^2}) 
\]
\[\Rightarrow [\sum \frac{1}{n^2} \text{ converges (Riemann series where } \alpha > 1\text{)} \Rightarrow \sum  e^{-\sqrt{n}} \text{ converges } ]
\]
\\
\\
{\ding{173}}\\

\text{Ex :} \\
\[
\sum ln(\frac{n+1}{n}) = \sum ln(1 + \frac{1}{n}) \Rightarrow ln(1+\frac{1}{n}) \underset{\,n \,\to \, +\infty}{\sim} \frac{1}{n}
\] \\
\[{1}{n}
\Rightarrow \sum ln(1+\frac{1}{n}) \text{ of same nature as} \sum \frac{1}{n} \Rightarrow \sum ln(1+\frac{1}{n}) \text{ diverges}
\]
\bigskip
\textcolor{red}{\subsubsection*{Definition :}}
\[
u_n \underset{\,+\infty}{\sim} v_n \Leftrightarrow \frac{u_n}{v_n} \underset{\,n \,\to\, +\infty}{\to} 1 \Rightarrow \exists \: n_0 \in \mathbb{N}, [n\ge n_0 \Rightarrow \text{sign of } u_n = \text{sign of } v_n] 
\] \\
\[
\Leftrightarrow \exists \: n_0 \in \mathbb{N}, [n \ge n_0 \Rightarrow u_n \times v_n \ge 0]
\]
\\
\textcolor{red}{Proposition :} \\

\[
\textcolor{red}{\text{Let } (u_n)_{n \,\in \,\mathbb{N}} \in \mathbb{R}^\mathbb{N}. \text{Then : }\sum(u_{n+1} -u_n) \text{ converges } \Leftrightarrow  (u_n)_{n\, \in\,\mathbb{N}} \text{ converges} }
\]\\

\textcolor{red}{Proof :}
\[
\text{Let } (S_n)_{n \,\in \,\mathbb{N}} \in \mathbb{R}^\mathbb{N} \text{ such that : }
\forall \, n \in \mathbb{N}, S_n = \sum_{\,k\,=\,0}^{n}{(u_{k+1} - {u_k}) = \sum_{\,k\,=\,0}^{n}{u_{k+1}} - \sum_{\,k\,=\,0}^{n}{u_{k}}} 
\]\\
\[
= \sum_{\,k\,=\,1}^{n+1}{u_{k}} - \sum_{\,k\,=\,0}^{n}{u_{k}} = u_{n+1} - u_{0} \Leftrightarrow \underset{\,n\, \to \,+\infty}{lim} S_n = \underset{\,n\, \to \,+\infty}{lim} (u_{n+1} - u_0)
\]\\
\bigskip	
\[
\implies [S_n \underset{\,n\,\to\,+\infty}{\to} l \Leftrightarrow u_n \underset{\,n\,\to\,+\infty}{\to} l\, +\,u_0\,]
\]
\newpage
\text{Ex :} \\

\[
\forall \, n\in \mathbb{N}^*, u_n = 1+ \frac{1}{2} +...+ \frac{1}{n} - ln(n) \hookleftarrow \text{nature of } (u_n)?
\] \\
\[
= \sum_{\,k\,=\,1}^{n}{\frac{1}{k}} - ln(n)
\]
\bigskip
\[
u_{n+1} -u_n = \sum_{\,k\,=\,1}^{n+1}{\frac{1}{k}} - ln(n+1) -\sum_{\,k\,=\,1}^{n}{\frac{1}{k}} - ln(n)
\]
\bigskip
\[
= \frac{1}{n+1} +ln(\frac{n}{n+1} = \frac{1}{n+1} + ln(1 -\frac{1}{n+1})
\]
\bigskip
\[
\frac{1}{n+1} - \frac{1}{n+1} - \frac{1}{2} \times \frac{1}{(n+1)^2} + o(\frac{1}{n^2}) = -\frac{1}{2} \times \frac{1}{(n+1)^2} + o(\frac{1}{n^2}) \underset{\,+\infty}{\sim} -\frac{1}{2} \times \frac{1}{n^2}
\]
\bigskip
\[
\implies \sum{u_{n+1} - u_n} \text{ converges} \Leftrightarrow u_n \text{ converges}
\]

\textcolor{red}{\subsection*{The Rules :}}

\textcolor{red}{\subsubsection*{Riemann's rule :}}
\bigskip
\textcolor{red}{Theorem :}\\

\textcolor{red}{Let ($u_n$) a positive term series :}
\bigskip

\textcolor{red}{If $\exists \: \alpha\in\mathbb{R}, \: \alpha>1, \: n^\alpha \times u_n \underset{\,n\,\to\,+\infty}{\to} 0$
then : $\sum{u_n}$ converges} 
\bigskip


\textcolor{red}{Proof :}\\

\[ n^\alpha \times u_n \, \underset{\,n\,\to\,+\infty}{\to} 0 \Leftrightarrow \frac{u_n}{\frac{1}{n^\alpha}} \: \underset{n\,\to \,+\infty}{\to} 0 \Leftrightarrow u_n = o(\frac{1}{n^\alpha})
\]
\bigskip

\text{Thus : [ $\sum{\frac{1}{n^\alpha}}$ converges (Riemann $\alpha > 1) \implies \sum u_n$ converges ]}
\newpage
\textcolor{red}{\subsubsection*{D'Alembert's rule :}}
\bigskip

\textcolor{red}{Theorem :}\\

\textcolor{red}{Let ($u_n$) a strictly positive sequence such that : }\\

\textcolor{red}{$\frac{u_{n+1}}{u_n} \: \underset{\,n \,\to \,+\infty}{\to} l$ where $l \in \mathbb{R} \:\cup \: \{ +\infty\}$}

\[
\textcolor{red}{ \text{Then : }
\begin{cases}
    l < 1 \Rightarrow \sum u_n \text{ converge} \\
    l > 1 \Rightarrow \sum u_n \text{ diverge}
\end{cases}
}
\]

\text{Ex : 
$\forall \, n\in \mathbb{N}, u_n = \frac{1}{n!}$}\\

\text{
$ \frac{u_{n+1}}{u_n} = \frac{1}{(n+1)!} \times n! = \frac{1}{n+1} \: \underset{\,n \, \to \, +\infty}{\to} 0 <1 \implies \sum{u_n}$ converges 
}\\
\bigskip

\textcolor{red}{\subsubsection*{Cauchy's rule :}}
\bigskip

\textcolor{red}{Theorem :}\\

\textcolor{red}{Let $u_n$ a strictly positive sequence such that :}

\[
\textcolor{red}{\sqrt[n]{u_n} \underset{\, n\, \to \, +\infty}{\to} l \text{ where l } \in \mathbb{R_+} \: \cup \:\{+\infty\}}
\] \\
\[
\textcolor{red}{\text{Then :} \begin{cases}
    \,l < 1 \implies \sum{u_n} \text{ converges}\\
    \, l< 1\implies \sum{u_n} \text{ diverges}
\end{cases}}
\]

\text{Ex :}
\[
\forall \, n\in\mathbb{N}^*, u_n = (\frac{n}{n+1})^{n^2} 
\]
\[
\sqrt[n]{u_n} = (u_n)^\frac{1}{n} = (\frac{n}{n+1})^n 
\text{ and we know that } (\frac{n}{n+1}) \underset{\, n\, \to \, +\infty}{\to} 1
\]
\[
\text{Moreover, } e^{nln(\frac{n}{n+1})} = e^{nln(1-\frac{1}{n+1})} 
\text{we have } n\times ln(1 - \frac{1}{n+1}) \underset{\,+\infty}{\sim} {-1} 
\]
\[
\implies e^{nln(\frac{n}{n+1})} \underset{\,+\infty}{\sim} {e^{-1}} 
\]

\newpage

\textcolor{red}{\subsection*{Series with Arbitrary terms :}}\bigskip

\textcolor{red}{\subsubsection*{Alternating Series}}
\bigskip

\textcolor{red}{Definition :}\\

\textcolor{black}{Let ($u_n$) be a real sequence. We say that ($u_n$) is alternating if there exist} \\

\textcolor{black}{a positive real sequence ($a_n$) such that $\forall \, n \in \mathbb{N}$ : }

\textcolor{red}{\[
u_n = (-1)^na_n \text{ or } u_n = (-1)^{n+1}a_n
\]} \\
\textcolor{red}{We say that a numerical series $\sum{u_n}$ is alternating if the sequence ($u_n$) is alternating}\bigskip
\bigskip
\\
\textcolor{red}{\textbf{Special criterion for alternating series :}}
\bigskip

\textcolor{red}{Leibniz Criterion :}\\

\text{Let ($u_n$) be an alternating real sequence.} \\

\text{
If ($|u_n|$) is decreasing and converges to 0, then :
} \\

\[
\textcolor{red}{\begin{cases}
\,\text{The series }\sum{u_n} \text{ converges} \\
\,\forall \, n \in \mathbb{N}, |R_n| \leq |u_{n+1}|
\end{cases}} \\
\]\\

\text{Where ($R_n$) is the sequence of remainders associated with $\sum{u_n}$}\\

\text{Ex :} \\

\text{Let $\alpha \in \mathbb{R}$. Then :} \\
\[
\sum{\frac{(-1)^n}{n^\alpha} \text{ converges if and only if } \alpha > 0}
\]

\newpage
\textcolor{red}{\subsubsection*{Absolute Convergence}}
\bigskip

\textcolor{red}{Definition :}\\

\textcolor{red}{A numerical series $\sum{u_n}$ is said to converge absolutely if the series $\sum{|u_n|}$} 

\textcolor{red}{ converges.} \\

\textcolor{red}{Theorem :} \\

\text{If $\sum{u_n}$ is a numerical series that converges absolutely, then $\sum{u_n}$ also converges} \\

\textcolor{red}{Definition :} \\

\textcolor{black}{A convergent series that is not absolutely convergent is called }

\textcolor{black}{conditionally convergent (or semi-convergent).}

\end{document}

