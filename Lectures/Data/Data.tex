\documentclass[a4paper,12pt]{article}
\usepackage{amsmath, amssymb}
\usepackage{xcolor}
\usepackage{ulem}        % underline
\usepackage{tcolorbox}   % colored boxes (if needed)
\usepackage{pifont}
\usepackage{xcolor}
\usepackage{tikz}
\usepackage{centernot}
\usepackage{booktabs}
\usetikzlibrary{trees} % for tree structures

\newcommand{\Eemp}{\hat{\mathbb{E}}}
\newcommand{\Varhat}{\widehat{\operatorname{Var}}}
\newcommand{\shat}{\hat{\sigma}}
\newcommand{\Covhat}{\widehat{\operatorname{Cov}}}
\newcommand{\chat}{\hat{c}}



\begin{document}

\begin{center}
\begin{center}
    \huge{\textbf{Mathématiques : CM}}
\end{center}
\end{center}
\newpage

\tableofcontents 


\newpage



\section{Euclidean space - scalar product}

\textbf{Idea : Let E a vector space of finite dimension} \bigskip

The scalar product is defined as :

\[u \in E, v \in E, <u, v>\]

\[< , > \; = \; \begin{cases} E \times E \; \rightarrow \; \mathbb{R} \\
    (u, v) \; \mapsto \; <u, v> 
    \end{cases} 
\] \bigskip

When u and v are orthogonal ($\perp$) we have the following relation :

\[ u \perp v \equiv <u, v> = 0
\]

We also have the following relation : \\

\[
||u|| = \sqrt{<u, u>}
\] \\

\textbf{Standard scalar product of $\mathbb{R}^3$ :} \\

\[
u = (x_1, y_1, z_1) \\
v = (x_2, y_2, z_2)
\]

\[<u,v> = x_1x_2 + y_1y_2 + z_1z_2\]
\[\implies ||u|| = \sqrt{<u,u>} = \sqrt{x_1^2 + y_1^2 + z_1^2}\] \\

\textbf{Definition : symmetric bilinear form} \\

Let a mapping such that :

\[
\phi : \begin{cases}
    E \times E \; \rightarrow \; \mathbb{R} \\
    (u, v)\; \mapsto \phi(u,v)
\end{cases}
\]

$\phi$ is a symmetric bilinear form if :

\begin{enumerate}
    \item It is bilinear : $\forall \; u_0 \in \, E, v \, \mapsto \, \phi (u_0, v)$ is a linear map
    \item It is bilinear : $\forall \; v_0 \in \, E, u \, \mapsto \, \phi (u, v_0)$ is a linear map
    \item It is symmetric : $\forall \; (u,v) \in \, E^2, \phi (u, v) = \phi(v,u)$ is a linear map
\end{enumerate} \bigskip

\textbf{Explanation :}

\[
\phi = \begin{cases}
    E \times E \; \rightarrow \, \mathbb{R} \\
    (u, v) \; \mapsto \phi (u,v)
\end{cases}
\]

Let $u_0 \in E$ : \\

The mapping : $\begin{cases}
    E \, \rightarrow \, \mathbb{R} \\
    v \, \mapsto \phi(u_0, v)
\end{cases}$is linear \\

\[
\forall \, (v, v') \in E^2, \forall \alpha \in \mathbb{R}, \phi(u_0, \alpha v + v') = \alpha \phi(u_0,v) + \phi(u_0, v')
\] \\

If $v_0 \in E, \forall (u,u') \in E^2,\forall \alpha \in \mathbb{R}, \phi(\alpha u + u', v_0) = \alpha \phi(u,v_0) + \phi(u', v_0) $ \\

\textbf{Example : standard scalar product of $\mathbb{R}^3$}

\[
u = (x_1, y_1, z_1) \\
v = (x_2, y_2, z_2)
\]

\[
\phi(u,v) = x_1x_2+ y_1y_2 + z_1z_2
\]

If u is given (considered as a constant), then the mapping  : \\

$v \, \mapsto \, x_1x_2 + y_1y_2 + z_1z_2$ is a linear map. \\

Note that the coordinates of u are constant numbers. \\ 

Those of v are variables.\\

for example if $u = (1,2,3)$, then $v = (x_2, 2y_2, 3z_2)$ \\

\textbf{Similarly :}, if $v = (x_2, y_2, z_2)$ is given, then the mapping : \\

$u \, \mapsto \, x_1x_2 + y_1y_2 + z_1z_2$ is a linear map . \\

Note that the coordinates of v are being constant numbers. \\

Those of u are variables. \\

\textbf{Furthermore}, $\phi(u,v) = \phi(v,u)$ \\

\textbf{A symmetric bilinear form $\phi$ is positive definite if :}

\[
\forall u \in E, \phi(u,u) \geq 0 \land \phi(u,u) = 0 \implies u = 0_E
\]

Example 1 : Standard scalar product of $\mathbb{R}^3$

\[
\phi(u,u) = x_1^2 + y_1^2 + z_1^2 \geq 0
\]

and \[
\phi(u,u) = 0 \implies \begin{cases}
    x_1^2 = 0 \\
    y_1^2 = 0 \\
    z_1^2 = 0 
\end{cases} \implies u = 0_E
\]

The standard scalar product is thus positive definite \\

Example 2 :

\[
\phi (u,v) = x_1x_2 + y_1y_2 - z_1z_2 \text{ is a symmetric bilinear form}
\] \bigskip

If $u = (x_1, y_1, z_1)$ is given (considered as a constant), then the mapping : \\

$v \, \mapsto \, x_1x_2 + y_1y_2 + z_1z_2$ is a linear map.\\

Note that the coordinates of u are constant numbers. \\

Those of v are the variables. \\

Thus the coordinates of u are the coefficient of the linear function. \\

Similarly, if v is given and constant, $u \mapsto \phi(u,v)$ is linear \\

\[\implies \phi(u,v) = \phi(v,u)\] \\

Positive definite ? No \\

\[
\phi(u,u) x_1^2 + y_1^2 - z_1^2
\] \bigskip

If $u = (1,0,0), \phi(u,u) = 1 > 0$ \\

If $u = (0,0,1), \phi(u,u) = -1 < 0$ \\

Example 3 :

\[
\phi (u,v) = x_1x_2 + y_1y_2
\phi(u,u) = x_1^2 + y_1^2 \geq 0
\]\bigskip

But $\phi(u,u) = 0 \centernot\implies u = 0_E$ as for $u=(0,0,1), \phi(u,u) = 0, u \neq 0_E$ \\

A scalar product is a symmetric bilinear form, which is positive definite. \\

Then we note $<u,v>$ instead of $\phi(u,v)$ \\

We can define for $u\in E$, 

\[
||u|| = \sqrt{<u,v>} \implies \sqrt{<u,v>} \in \mathbb{R} \text{ because } <u,v> \; \geq 0
\]

\[
||u|| = 0 \implies u = 0_E
\] \bigskip

Theorem : Matrix representation of a symmetric bilinear form :\\

Let $\mathcal{B} = (e_1, ..., e_n)$ be a basis of E. \\

u has coordinates : $\begin{pmatrix}
    x_1 \\
    y_1 \\
    z_1 \\
    ... \\
\end{pmatrix}$ in $\mathcal{B}$ \bigskip


v has coordinates : $\begin{pmatrix}
    x_2 \\
    y_2 \\
    z_2 \\
    ... \\
\end{pmatrix}$ in $\mathcal{B}$ \bigskip

Then : $\exists \, A \in \mathbb{M}_n(\mathbb{R}), \phi(u,v) = \,^+{X}_1AX_2$ is symmetric \\

Examples : \\

Standard scalar product of $\mathbb{R}^3$ : \\

Example 1 : \\

$\mathcal{B}$ standard basis of $\mathbb{R}^3$ \\

$u = (x_1, y_1, z_1)$ has coordinates $X_1 = \begin{pmatrix}
    x_1 \\
    y_1 \\
    z_1 \\
\end{pmatrix}$ \bigskip

$v = (x_2, y_2, z_2)$ has coordinates $X_2 = \begin{pmatrix}
    x_2 \\
    y_2 \\
    z_2 \\
\end{pmatrix}$ \bigskip

\[
^T{X_1}X_2 = (x_1, y_1, y_2) \begin{pmatrix}
    x_2\\
    y_2\\
    z_2\\
\end{pmatrix}
\]\bigskip

\[
= x_1x_2 + y_1y_2 +z_1z_2 \implies A = I_3 = \begin{pmatrix}
    1 & 0 & 0 \\
    0 & 1 & 0 \\
    0 & 0 & 1
\end{pmatrix}
\]

\[
^T{X_1}AX_2 = x_1x_2 + y_1y_2 + z_1z_2 = <u,v>
\] \bigskip

Example 2 :

\[
\phi(u,v) = x_1x_2 + y_1y_2 - z_1z_2
\]

\[\text{Then : } A = \begin{pmatrix}
    1 & 0 & 0 \\
    0 & 1 & 0 \\
    0 & 0 & -1
\end{pmatrix}\] \bigskip

\[^T{X_1}AX_2 = (x_1, y_1, z_1) \begin{pmatrix}
    1 & 0 & 0 \\
    0 & 1 & 0 \\
    0 & 0 & -1
\end{pmatrix} \begin{pmatrix}
    x_2 \\
    y_2 \\
    z_2
\end{pmatrix}\]\bigskip

\[
= (x_1, y_1, z_1) \begin{pmatrix}
    x_2 \\
    y_2 \\
    -z_2
\end{pmatrix}
\] 

\[
= x_1x_2 + y_1y_2 - z_1z_2
\]

Example 3 : 

\[
\phi(u,v) = x_1x_2 + y_1y_2
\]

\[ \text{Then : } A = \begin{pmatrix}
    1 & 0 & 0 \\
    0 & 1 & 0 \\
    0 & 0 & 0
\end{pmatrix}\] \bigskip

\[^T{X_1}AX_2 = (x_1, y_1, z_1) \begin{pmatrix}
    1 & 0 & 0 \\
    0 & 1 & 0 \\
    0 & 0 & 0
\end{pmatrix} \begin{pmatrix}
    x_2 \\
    y_2 \\
    z_2
\end{pmatrix}\] \bigskip

\[
= (x_1, y_1, z_1) \begin{pmatrix}
    x_2 \\
    y_2 \\
    0
\end{pmatrix}
\]

\[
= x_1x_2 + y_1y_2
\]
\bigskip

\subsection{Spectral Theorem} 
\bigskip

\textbf{Spectral Theorem :} all symmetric matrix A is diagonalizable. \\

We can choose an orthonormal eigenbasis ($\epsilon_1, ..., \epsilon_n$). \\

Then the transition matrix P satisfies to :

\[
^TPP = P \times {^T}P = I \Leftrightarrow P^{-1} = {^T}P
\] \\

Note : if the eigenbasis is orthonormal, then : 

\[
<\epsilon_i, \epsilon_j> = \begin{cases}
    1 \text{ if }i = j \\
    0 \text{ otherwise}
\end{cases}
\] \\

\subsection{Theorem} 
\bigskip

\textbf{Theorem :} Let E = $\mathbb{R}^n$, let A be a symmetric matrix in $\mathcal{M}_n(\mathbb{R})$. \\

Let us define $\phi(u,v) = {^T}X_1 A X_2$. \\

$\phi$ is a scalar product $\Leftrightarrow$ All eigenvalues of A are > 0.\\

Explanation in a an orthogonal eigenbasis $\mathcal{B'}$, $\phi(u,v) = {^TX'_1}DX'_2$ \\

Where D is a diagonal matrix.

\newpage 

\section{Orthogonal Projection}

Let $V_0$ be the orthogonal projection of $V$ onto $F$. \bigskip

$V_0$ is the element of $F$ the closest to $V$. \\

\subsection{Motivation}

Assume you must find $\lambda \in \mathbb{R}$ such that :

\[
v = \lambda u
\]

The "best solution" is $\lambda$ such that $v_0 = \lambda u$ \\

Ex: $u = (1,1), v = (2,1)$ \\

Solving $v = \lambda u : \begin{cases}
    \lambda = 2 \\
    \lambda = 1
\end{cases}$ \\

\[
v_0 = \alpha u = (\alpha, \alpha)
\]
\[(v - v_0) \perp u\]
\[(2 - \alpha, 1 - \alpha) \perp (1,1)\]
\[\implies (2 - \alpha) \times 1 + (1 - \alpha) \times 1 = 0\]
\[\implies 3 - 2\alpha = 0 \implies \alpha = \frac{3}{2}\]
\[\implies v_0 = \frac{3}{2}(1,1) = (\frac{3}{2}, \frac{3}{2})\]\\

The solution does not have a solution.\\

However, we can determine $v_0$.\\

$v_0$ represent the solution with the samellest range of error.

\subsection{Definition}

Let $(E, <, >)$ be an euclidean space. \\

Let $F$ be a linear subspace of E.

\[
F^\perp = \{ u \in E, \forall \, v \in F, <u,v> = 0 \}
\]

\subsection{Theorem}

If $F$ admits a basis $\mathcal{B} = (e_1, e_2, ..., e_n)$, then $F^\perp = \mathcal{B}^\perp$

\[
F^\perp = \{ u \in E, \begin{cases}
    <u, e_1> = 0 \\
    <u, e_2> = 0 \\
    ... \\
    <u, e_n> = 0
\end{cases}\}
\]

\subsection{Theorem}

If $F$ is finite-dimensional, then $F \oplus F^\perp = E$

\[
\Leftrightarrow \forall \, u \in E, \exists! (v,w) \in F \times F^\perp, u = v + w
\] \\

Then $v$ is the orthogonal projection of $u$ onto $F$.

\subsection{Property}

Let $u \in E$ and let $v_0 = P_F(u)$. \\

$P_F$ is the orthogonal projection onto $F$. \\

$v_0$ is the element of $F$ which minimizes $||u - v||^2$.\\

It is the solution of the optimization : \\

\[
Min(||u - v||), \text{ constraint : } v \in F
\]

\subsection{Proof}

Let $u = v_0 + w$ where $(v_0,w) \in F \times F^\perp$. \\

Then $\forall \, v \in F$ :

\[
||u - v||^2 = <u-v, u-v> \implies u - v = (u-v_0) + (v_0 - v)
\] \\

Indeed, $(u-v_0) = w \in F^\perp$ and $(v_0 - v) \in F$ as $(v_0, v) \in F^2$

\[
\implies ||u - v||^2 - <u-v, u-v> = <(u-v_0) + (v_0 - v), (u-v_0) + (v_0 - v)> 
\]

\[
= \,<(u - v_0), (u - v_0)> + \; 2 <(u - v_0), (v_0 - v)>  + <(v_0 - v), (v_0 - v)> 
\]

\[
= ||u - v_0||^2 + ||v_0 - v||^2 \geq ||u - v_0||^2
\]

\[
P_F(u) \in F \Leftrightarrow \exists (\lambda_1, ..., \lambda_n) \in \mathbb{R}^n, u = \lambda_1e_1 + ... + \lambda_ne_n
\]

\[
u - P_F(u) \in F^\perp \Leftrightarrow \begin{cases}
    <(u - \lambda_1e_1 + ... + \lambda_ne_n), e_1> = 0 \\
    ... \\
    <(u - \lambda_1e_1 + ... + \lambda_ne_n), e_n> = 0
\end{cases}
\] \\

\[
\Leftrightarrow \begin{cases}
    <(\lambda_1<e_1,e_1> + ... + \lambda_n<e_n,e_n>), e_1> = <u,e_1> \\
    ... \\
    <(\lambda_1<e_1,e_1> + ... + \lambda_n<e_n,e_n>), e_n> = <u, e_n>
\end{cases}
\] \\

\subsection{Finding $P_F(u)$}

Give $F$ admits a basis $\mathcal{B_F} = (\epsilon_1, ..., \epsilon_n)$ \\

We write $P_F(u) \in F \implies u - P_F(u) \in F^\perp$

\[
\implies u = P_F(u) + (u - P_F(u))
\]

\[
P_F(u) \in F \Leftrightarrow \exists (\alpha_1, ..., \alpha_n) \in \mathbb{R}^n, P_F(u) = \alpha_1\epsilon_1 + ... +  \alpha_n\epsilon_n
\]

\[
u - P_F(u) \in F^\perp \Leftrightarrow \begin{cases}
    <u - P_F(u), \epsilon_1> = 0 \\
    ... \\
    <u - P_F(u), \epsilon_n> = 0
\end{cases}
\] \\

\[
\Leftrightarrow \begin{cases}
    <u - \alpha_1\epsilon_1 + ... + \alpha_n\epsilon_n, \epsilon_1> = 0 \\
    ... \\
    <u - \alpha_1\epsilon_1 + ... + \alpha_n\epsilon_n, \epsilon_n> = 0
\end{cases}
\] \\

\[
\Leftrightarrow \begin{cases}
    <\alpha_1 <\epsilon_1, \epsilon_1> + ... + \alpha_n <\epsilon_n, \epsilon_n>, \epsilon_1> = <u, \epsilon_1> \\
    ... \\
    <\alpha_1 <\epsilon_1, \epsilon_1> + ... + \alpha_n <\epsilon_n, \epsilon_n>, \epsilon_n> = <u, \epsilon_n>
\end{cases}
\] \\
\newpage

\section{Data Analysis} \bigskip

Example of a dataset : \newline

\begin{table}[h]
\centering
\begin{tabular}{l c c c c c c c}
\toprule
Student & Class & Average & Sport & French & Math & Physics & English \\
\midrule
Student1 & A1 & 1 & SP1 & FR1 & MATH1 & PHY1 & ENG1 \\
Student2 & B2 & 1 & SP2 & FR2 & MATH2 & PHY2 & ENG2 \\
Student3 & C1 & 1 & SP3 & FR3 & MATH3 & PHY3 & ENG3 \\
Student4 & B2 & 1 & SP4 & FR4 & MATH4 & PHY4 & ENG4 \\
\bottomrule
\end{tabular}
\caption{Table of marks average in a high school.}
\label{tab:Dataset}
\end{table}
\bigskip
The question that we adress:
\begin{itemize}
    \item Consider the student good in maths. Are they good in physics ?
\end{itemize}

We try to find a relation :

\[
    PHY = \alpha MAT + \beta + Err_1
\] 

And compare the error with :

\[
    PHY = \beta + Err_2
\] 

The "best" $\beta$ is the Average of PHY column. \newline

Is $||Err_1|| <$ $||Err_2||$ ? \newline

If yes, is $\alpha > 0$ ? \newline

For the model : 
\[
    PHY = \alpha MAT + \beta + Err_1
\] 

we look for the best solution of : 

\[
(S) : \begin{cases}
    PHY_1 = \alpha MAT_1 + \beta \\
    PHY_2 = \alpha MAT_2 + \beta \\
    ... \\
\end{cases}
\]

\[
    \Leftrightarrow PHY = \alpha MAT + \beta \begin{pmatrix}
        1 \\
        1 \\
        ... \\
        1 \\
    \end{pmatrix}
\]

Where $PHY$ and $MAT$ are the whole column. \newline

Optimal values : $\alpha_0$ and $\beta_0$ such that $\alpha_0 MAT + \beta_0 \begin{pmatrix}
        1 \\
        1 \\
        ... \\
        1 \\
    \end{pmatrix}$ is the orthogonal \newline 
    
    projection of $PHY$ on $Span(Mat, \begin{pmatrix}
        1 \\
        1 \\
        ... \\
        1 \\
    \end{pmatrix})$ \newline

\begin{itemize}
    \item The students good at maths and sport, do they tend to be even better than those good only at maths ?
\end{itemize}

\[
    PHY = \alpha MAT + \beta\begin{pmatrix}
        1 \\
        1 \\
        ... \\
        1 \\
    \end{pmatrix} + \gamma SP + Err_3
\]

Find $\alpha_0, \beta_0, \gamma_0$ optimal : \newline

Is $||Err_3|| < ||Err_1||$ ? \newline

If yes, sign of $\beta$ ? \newline

Let X be a column of the table.

\[
    X = \begin{pmatrix}
        x_1 \\
        x_2 \\
        ... \\
        x_n 
    \end{pmatrix}
\]

X is an identically distributed random variable and independant. \newline

Empirical expectation of X : \newline

\[
\Eemp (X) = \frac{x_1 + x_2 + ... + x_n}{n} = \frac{^T\begin{pmatrix}
    1 \\
    ... \\
    1 \\
\end{pmatrix}.X}{n}
\] \newline

Empirical variance of X : \newline

\[
\Varhat (X) = \Eemp((X - \Eemp (X))^2)
\]

\[
= \frac{(x_1 - \Eemp(X))^2 + (x_2 - \Eemp(X))^2 + ... + (x_n - \Eemp(X))^2}{n}
\] \newline

Let $X_c$ = $X - \Eemp (X) \begin{pmatrix}
    1 \\
    ... \\
    1 \\
\end{pmatrix}$\newline

\[
X_c = \begin{pmatrix}
    x_1 - \Eemp (X)\\
    x_2 - \Eemp (X)\\
    \dots
    x_n - \Eemp (X)\\
\end{pmatrix}
\] 

$X_c$ is the centered version of X. \newline

\[
\text{Thus } \Varhat = \frac{(x_1 - \Eemp(X))^2 + (x_2 - \Eemp(X))^2 + ... + (x_n - \Eemp(X))^2}{n}
\]

\[
= \frac{1}{n} \begin{pmatrix}
    x_1 - \Eemp(X) & x_2 - \Eemp(X) & ... & x_n - \Eemp(X)
\end{pmatrix} \begin{pmatrix}
    x_1 - \Eemp(X) \\
    x_2 - \Eemp(X) \\
    ... \\
    x_n - \Eemp(X)
\end{pmatrix}
\]

\[
= \frac{^TX_cX_c}{n} = \frac{||X_c||^2}{n}
\] \newline

Empirical standard deviation of X :

\[
\shat(X) = \sqrt{\Varhat(X)} = \sqrt{\frac{^TX_cX_c}{n}} = \frac{||X_c||}{\sqrt{n}}
\] \newline

Consider two columns $X = \begin{pmatrix}
    x_1 \\
    ... \\
    x_n \\
\end{pmatrix}$ and $Y = \begin{pmatrix}
    y_1 \\
    ... \\
    y_n \\
\end{pmatrix}$ \newline

Empirical covariance of X and Y :

\[
\Covhat(X,Y) = \Eemp((X - \Eemp(X))\times(Y - \Eemp(Y)))
\]
\[
= \frac{1}{n}(x_1 - \Eemp(X))(y_1 - \Eemp(Y)) + ... + (x_n - \Eemp(X))(y_n - \Eemp(Y))
\]
\[
= \frac{1}{n} \begin{pmatrix}
    x_1 - \Eemp(X) & x_2 - \Eemp(X) & ... & x_n - \Eemp(X)
\end{pmatrix} \begin{pmatrix}
    y_1 - \Eemp(Y) \\
    y_2 - \Eemp(Y) \\
    ... \\
    y_n - \Eemp(Y)
\end{pmatrix}
\]
\[
= \frac{^TX_cY_c}{n}
\]
\[
\frac{1}{n}<X_c, Y_c>
\] \newline

Empirical correlation coefficient of X and Y : 

\[
\chat (X, Y) = \frac{\Covhat(X,Y)}{\shat(X)\shat(Y)}
\] \newline

Note that $\Covhat(X, X) = \Varhat(X)$ and $\Covhat(Y, Y) = \Varhat(Y)$ \newline

Let $D =$ Dataset = $\begin{pmatrix}
    X & Y
\end{pmatrix} = \begin{pmatrix}
    x_1 & y_1 \\
    \dots & \dots \\
    x_n & y_n
\end{pmatrix}$ \newline

$D_c =$ centered data set $= \begin{pmatrix}
    X_c & Y_c
\end{pmatrix}$ \newline

The Empirical covariance matrix of D is : 

\[
\Covhat(D) = \begin{pmatrix}
    \Varhat(X) & \Covhat(X, Y) \\
    \Covhat(X, Y) & \Varhat(Y)
\end{pmatrix} = \frac{^TD_cD_c}{n}
\]

\[
^TD_cD_c = \begin{pmatrix}
    x_1 - \Eemp(X) & ... & x_n - \Eemp(X) \\
    y_1 - \Eemp(Y) & ... & y_n - \Eemp(Y)
\end{pmatrix} \begin{pmatrix}
     x_1 - \Eemp(X) & y_1 - \Eemp(Y) \\
     ... & ... \\
     x_n - \Eemp(X) & y_n - \Eemp(Y)
\end{pmatrix}
\]

\[
= \begin{pmatrix}
    ^TX_cX_c & ^TX_cY_c \\
    ^TY_cX_c & ^TY_cY_c \\
\end{pmatrix}
\]

Ex : find $\Varhat(aX + bY)$ where $aX + bY = D \begin{pmatrix}
    a \\
    b
\end{pmatrix}$ \newline

\[
D \begin{pmatrix}
    a \\
    b
\end{pmatrix} = \begin{pmatrix}
    x_1 & y_1 \\
    x_2 & y_2 \\
    \dots & \dots \\
    x_n & y_n \\
\end{pmatrix} \begin{pmatrix}
    a \\
    b
\end{pmatrix} = \begin{pmatrix}
    ax_1 + by_1 \\
    ax_2 + ny_2 \\
    \dots \\
    ax_n + by_n \\
\end{pmatrix} = aX + bY
\]

\[
\Varhat(X) = {^T}{X_c}X_c
\]

\[
\Varhat(D\begin{pmatrix}
    a \\
    b
\end{pmatrix}) = (D_c\begin{pmatrix}
    a \\
    b
\end{pmatrix}) (D_c\begin{pmatrix}
    a \\
    b
\end{pmatrix}) \times \frac{1}{n}
\]

\[
= \begin{pmatrix}
    a & b
\end{pmatrix} {^T}D_cD_c \begin{pmatrix}
    a \\
    b
\end{pmatrix} \times \frac{1}{n}
\]
\[
= \begin{pmatrix}
    a & b
\end{pmatrix} \Covhat(D) \begin{pmatrix}
    a \\
    b
\end{pmatrix} 
\]
\end{document}