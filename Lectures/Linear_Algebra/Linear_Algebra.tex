\documentclass[a4paper,12pt]{article}
\usepackage{amsmath, amssymb}
\usepackage{xcolor}
\usepackage{ulem}        % underline
\usepackage{tcolorbox}   % colored boxes (if needed)
\usepackage{pifont}
\usepackage{xcolor}
\usepackage{tikz}
\usetikzlibrary{trees} % for tree structures

\begin{document}
\begin{center}
\textbf{Mathématiques : CM}
\end{center}


\section{Bases of a Vector Space}

\subsection*{Definition 1}
Let $E$ be a vector space over a field $K$, let $n \in \mathbb{N}$, and let $\{x_1, x_2, \ldots, x_n\}$ be a family of vectors in $E$.  
We say that the family $\{x_1, x_2, \ldots, x_n\}$ is a \textbf{basis} of $E$ if it is both \textbf{linearly independent} and \textbf{spanning} (generating) for $E$.

\subsection*{Theorem 1 (Characterization of a Basis)}
Let $E$ be a vector space over $K$, let $n \in \mathbb{N}$, and let $\{x_1, x_2, \ldots, x_n\}$ be a family of vectors in $E$.  
Then $B = \{x_1, x_2, \ldots, x_n\}$ is a basis of $E$ if and only if:
\[
\forall x \in E, \ \exists! (\lambda_1, \lambda_2, \ldots, \lambda_n) \in K^n \text{ such that } 
x = \sum_{i=1}^{n} \lambda_i x_i.
\]
The scalars $\lambda_i$, $i \in \{1, 2, \ldots, n\}$ are called the \textbf{coordinates} of the vector $x$ in the basis $B$.

\section{Existence of a Basis}

\subsection*{Theorem 2}
Every non-zero vector space $E$ over a field $K$ admits at least one basis.

\subsection*{Theorem 3 (Incomplete Basis Theorem)}
Let $E$ be a vector space over $K$ that admits a finite generating family. Then:
\begin{enumerate}
    \item From every generating family of $E$, one can extract a basis of $E$.
    \item Every linearly independent family of $E$ can be extended to a basis of $E$.
\end{enumerate}

\subsection*{Definition 2}
In $\mathbb{R}^n$, the family $(e_1, \ldots, e_n)$ where for every $i \in \{1, \ldots, n\}$,
\[
e_i = (0, \ldots, 0, \underbrace{1}_{\text{$i$-th coordinate}}, 0, \ldots, 0)
\]
is a basis. It is called the \textbf{canonical basis} of $\mathbb{R}^n$.

\subsection*{Definition 3}
In $\mathbb{R}_n[X]$, the family $(1, X, \ldots, X^n)$ is a basis.  
It is called the \textbf{canonical basis} of $\mathbb{R}_n[X]$.

\section{Dimension of a Vector Space}

\subsection*{Definition 1}
Let $E$ be a vector space over a field $K$.  
We say that $E$ is of \textbf{finite dimension} if $E$ admits a finite generating family.

\subsection*{Theorem 1}
Let $E \neq \{0_E\}$ be a finite-dimensional vector space over $K$. Then:
\begin{enumerate}
    \item $E$ admits at least one basis.
    \item All bases of $E$ have the same cardinality.
\end{enumerate}

\subsection*{Definition 2}
Let $E$ be a finite-dimensional vector space over $K$.
\begin{enumerate}
    \item If $E \neq \{0_E\}$, we call the \textbf{dimension} of $E$, denoted $\dim(E)$, the cardinality of any basis of $E$.
    \item If $E = \{0_E\}$, we define $\dim(E) = 0$.
\end{enumerate}

\bigskip
\noindent \textbf{Examples:}  
$\dim(\mathbb{R}^n) = n$ and $\dim(\mathbb{R}_n[X]) = n + 1$.

\subsection*{Proposition}
Let E a $\mathbb{K}$ vector space such that dim(E) = n, n$\in \mathbb{N*}$ and B a family of p vectors, $p \in \mathbb{N*}$.
\bigskip

Then : 
\begin{enumerate}
    \item B is linearly independant $\implies p \leq n$
    \item B is a spanning family of E $\implies p \geq n$
    \item B is a basis of E $\implies (p \leq n) \land (p \geq n) \Leftrightarrow p = n$
    \item (p = n) $\implies (B \text{ is linearly independant})$
\end{enumerate}
\bigskip
\noindent \textbf{Examples:}  
We have $\{1, X - 1, (X + 1)^2 \}$ = $\mathbb{R}_2[X]$ and $dim(\mathbb{R}_2[X]) = 3$

\bigskip
Then : \begin{enumerate}
    \item $Card(\{1, X - 1, (X + 1)^2 \}) = 3 = dim(\mathbb{R}_2[X])$
    \item B is linearly independant
\end{enumerate}

Proof that B is linearly independant : 
\[
\begin{pmatrix}
1 & -1 & 1 \\
0 & 1 & 2 \\
0 & 0 & 1
\end{pmatrix}
:
\begin{pmatrix}
    0 \\
    0 \\
    0
\end{pmatrix}
\implies a = b = c = 0 \implies a \,+ \, b \, \times \, (X - 1) + c \,\times\,(X+1)^2 = 0_{\mathbb{R}_2[X]}
\]

\bigskip
\noindent \textbf{Examples:} 

$F = span(u_1 =, u_2, u_3), \forall i \in [\, | 1, 3 | \,], \forall \, u_i \text{ from } \mathbb{R^3}$
\bigskip

Find values of t such that $dim(F) = 3$, where :

\[
u_1 = (0, t^2, 1)
\]
\[
u_2 = (0, 0, 3)
\]
\[
u_3 = (1, 1, 3)
\]

$dim(F) = 3 \Leftrightarrow F = E $ because F is a linear subspace of E.
\bigskip

We are looking for a value of B such that $span(B) = \mathbb{R}^3 \implies$ we are looking for t such that B is a basis of $\mathbb{R}^3$
since $Card(B) = 3 = dim(E) \implies$ we are looking for t such that B is linearly independant.

\[\Leftrightarrow
\begin{pmatrix}
    0 & 0 & 1 \\
    t^2 & 0 & 1 \\
    1 & 3 & 3 
\end{pmatrix}
:
\begin{pmatrix}
    0 \\
    0 \\
    0
\end{pmatrix}
\Leftrightarrow
\begin{pmatrix}
    1 & 3 & 3 \\
    t^2 & 0 & 1 \\
    0 & 0 & 1 
\end{pmatrix}
:
\begin{pmatrix}
    0 \\
    0 \\
    0
\end{pmatrix}
\Leftrightarrow
\begin{pmatrix}
    1 & 3 & 3 \\
    0 & -3t^2 & 1 -3t^2 \\
    0 & 0 & 1 
\end{pmatrix}
:
\begin{pmatrix}
    0 \\
    0 \\
    0
\end{pmatrix}
\]

Case t = 0 :

$(a_1, a_2, a_3) \neq (0, 0, 0)$

\bigskip
Case t $\neq$ 1 :

Kramer system : $a_1 = a_2 = a_3 = 0$

\section{Linear Maps}

\subsection*{Definition :}
Let A,B two $\mathbb{K}$ vector space and 
\[
f : \begin{cases}
    A \to B \\
    \alpha \mapsto f(x)
\end{cases}
\]

$\alpha$ map from A to B. Then :

\[
f \text{ is linear} \Leftrightarrow \begin{cases}
    \forall (\alpha, u) \in \mathbb{K} \times A, f(\alpha u) = \alpha f(u) \\
    \forall (u, v) \in A^2, f(u + v) = f(u) + f(v)
\end{cases}
\]
\[
\Leftrightarrow \forall (\alpha, u, v) \in \mathbb{K} \times A^2, f(\alpha u + v) = \alpha f(u) + f(v)
\]

Notation : f is a linear map from A to B $\Leftrightarrow f \in \mathcal{L}(A,B)$ \\

\textbf{Remark (necessary condition) :} $f \in \mathcal{L}(A,B)\implies f(0_A) = O_B$ \\

\textbf{Contrapositive :} $f(0_A) \neq O_B \implies f \notin \mathcal{L}(A,B)$

\subsection*{Definition :}

Let $f \in \mathcal{L}(A,B)$ with A and B two $\mathbb{K}$ vector space. Then :

\begin{itemize}
    \item We call $Ker(f) = \{ X \in A, f(X) = 0_B\} \subset A$
    \item We call $Im(f) = \{Y \in B, f(X) = Y \} = \{f(X), X \in A\} \subset B$
\end{itemize}

\subsection*{Proposition :}

\begin{enumerate}
    \item $Ker(f)$ is a linear subspace of A
    \item $Im(f)$ is a linear subspace of B
\end{enumerate}

\subsection*{Definition :}

Let $f \in \mathcal{L}(A,B)$.
\[
\begin{cases}
    f \text{ is injective} \Leftrightarrow Ker(f) = \{0_A\} \\
    f \text{ is surjective} \Leftrightarrow Im(f) = B
\end{cases}
\]

\subsection*{Proof :}

\[f \text{ is surjective} \implies \forall Y \in B, \exists \, x \in A, f(X) = Y\]
\[\implies B \subset Im(f) \land Im(f) \subset B \text{ by definition} \implies Im(f) = B\]

\end{document}

